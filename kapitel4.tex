%
%	Praxisbezug
%

\pagebreak
\section{Data Analysis}

\onehalfspacing

\subsection{Performance Considerations}

In the previous chapters, we've already identified our key performance metrics: CPU consumption, memory usage, and TCP bandwidth. 

In the first step of the analysis, we will tabulate the performance metrics and, from the results, define a ranking for each CNI within each metric and each year.

As a final step of the analysis, we will weigh the rankings to arrive at an overall ranking for the CNI performance and guidance on which CNI to use.

\subsection{Server CPU Usage}

The first metric that we want to look at is server CPU usage.

\begin{table}[H]
\caption{Median Server CPU Usage}
\begin{tabular}{|c | c | c | c | c|} 
 \hline
 Year & Flannel & Calico & Canal & Cilium \\
 \hline\hline
 2020 & 5.05 & 6.44 & 6.69 & 13.22 \\ 
 \hline
 2021 & 5.01 & 6.36 & 6.66 & 12.77 \\
 \hline
 2024 & 5.05 & 5.44 & 5.64 & 6.15 \\
 \hline
\end{tabular}
\label{tab:cpu}
\end{table}

We get a reasonably uniform distribution across the three measurements. In 2024, we have a much more powerful set of hardware for our servers with more modern CPUs, and we can see that Cilium benefits the most from the increase in CPU speed.

Overall, we get the following ranking of the CNIs:

\begin{table}[H]
\caption{Server CPU Usage Ranking}
\begin{tabular}{|c | c | c | c | c|} 
 \hline
 Year & 1st & 2nd & 3rd & 4th \\
 \hline\hline
 2020 & Flannel & Calico & Canal & Cilium \\ 
 \hline
 2021 & Flannel & Calico & Canal & Cilium \\
 \hline
 2024 & Flannel & Calico & Canal & Cilium \\
 \hline
\end{tabular}
\label{tab:cpu-r}
\end{table}

\subsection{Server Memory Consumption}

The second metric we want to analyze is server memory consumption.

\begin{table}[H]
\caption{Median Server Memory Consumption}
\begin{tabular}{|c | c | c | c | c|} 
 \hline
 Year & Flannel & Calico & Canal & Cilium \\
 \hline\hline
 2020 & 590 & 659 & 655 & 866 \\ 
 \hline
 2021 & 590 & 661 & 659 & 867 \\
 \hline
 2024 & 1175 & 1420 & 1285 & 1521 \\
 \hline
\end{tabular}
\label{tab:mem}
\end{table}

We again get a reasonably uniform distribution, albeit with Canal in 2nd place instead of Calico. The 2024 data shows that all CNIs use the more powerful hardware and consume more memory across the board.

This leads us to the following ranking of the CNIs:

\begin{table}[H]
\caption{Server Memory Consumption Ranking}
\begin{tabular}{|c | c | c | c | c|} 
 \hline
 Year & 1st & 2nd & 3rd & 4th \\
 \hline\hline
 2020 & Flannel & Canal & Calico & Cilium \\ 
 \hline
 2021 & Flannel & Canal & Calico & Cilium \\
 \hline
 2024 & Flannel & Canal & Calico & Cilium \\
 \hline
\end{tabular}
\label{tab:mem-r}
\end{table}

\subsection{Pod-to-Pod Bandwidth}

The third metric we'll analyze will be the first bandwidth metric, TCP Pod-to-Pod traffic.

\begin{table}[H]
\caption{Median Pod-to-Pod Bandwidth}
\begin{tabular}{|c | c | c | c | c|} 
 \hline
 Year & Flannel & Calico & Canal & Cilium \\
 \hline\hline
 2020 & 9705 & 8882 & 8634 & 9475 \\ 
 \hline
 2021 & 9695 & 8876 & 8612 & 9444 \\
 \hline
 2024 & 19166 & 18571 & 16842 & 20709 \\
 \hline
\end{tabular}
\label{tab:p2pbw}
\end{table}

The bandwidth distribution is slightly more varied, with Cilium gaining the most over the years. The 2024 data shows that all CNIs use the faster NIC speed and the updated kernel, and an overall significant increase in bandwidth can be observed.

We get the following ranking of the CNIs, with Cilium overtaking Flannel and moving to first place:

\begin{table}[H]
\caption{Pod-to-Pod Bandwidth Ranking}
\begin{tabular}{|c | c | c | c | c|} 
 \hline
 Year & 1st & 2nd & 3rd & 4th \\
 \hline\hline
 2020 & Flannel & Cilium & Calico & Canal \\ 
 \hline
 2021 & Flannel & Cilium & Calico & Canal \\
 \hline
 2024 & Cilium & Flannel & Calico & Canal \\
 \hline
\end{tabular}
\label{tab:p2pbw-r}
\end{table}

\subsection{Pod-to-Server Bandwidth}

The fourth and final metric will be the second bandwidth metric, TCP Pod-to-Server traffic.

\begin{table}[H]
\caption{Median Pod-to-Server Bandwidth}
\begin{tabular}{|c | c | c | c | c|} 
 \hline
 Year & Flannel & Calico & Canal & Cilium \\
 \hline\hline
 2020 & 9828 & 8675 & 8576 & 9673 \\ 
 \hline
 2021 & 9825 & 8763 & 8579 & 9679 \\
 \hline
 2024 & 19942 & 19263 & 16328 & 21758 \\
 \hline
\end{tabular}
\label{tab:p2ebw}
\end{table}

The data again shows that in 2024, all CNIs use the faster NIC speed and show an overall increase in bandwidth.

We get the same ranking for the CNIs, with Cilium overtaking Flannel and again occupying first place in TCP Pod-to-Server communication:

\begin{table}[H]
\caption{Pod-to-Server Bandwidth Ranking}
\begin{tabular}{|c | c | c | c | c|} 
 \hline
 Year & 1st & 2nd & 3rd & 4th \\
 \hline\hline
 2020 & Flannel & Cilium & Calico & Canal \\ 
 \hline
 2021 & Flannel & Cilium & Calico & Canal \\
 \hline
 2024 & Cilium & Flannel & Calico & Canal \\
 \hline
\end{tabular}
\label{tab:p2ebw-r}
\end{table}

\subsection{Weighted Ranking}

Now that we have the individual performance rankings for our key indicators, we must identify the best overall choice. To do this, we'll use a weighted ranking of the performance indicators to arrive at a final result.

The weighted ranking works by assigning a rank or score to individual items based on multiple criteria, where each criterion is given a specific weight or importance level.\footnote{See \textit{Gemini (2024)}: Weighted Ranking. \cite{bardWeigthed}}

To judge overall network performance, we'll assign the following weights to the rankings:

\begin{itemize}
    \item CPU usage: 20\%
    \item Memory consumption: 30\%
    \item Pod-to-Pod Bandwidth: 25\%
    \item Pod-to-Server Bandwidth: 25\%
\end{itemize}

To calculate the weighted rankings, we'll use the reverse values for the individual rankings of the four CNIs, i.e., 4 for first place and 1 for last place, from the tables above.

We'll use Microsoft Excel to tabulate the results.\footnote{See \textit{Bobbitt, Z. (2023)}: How to Calculate Weighted Ranking in Excel. \cite{howtoWeigh}}

\subsection{Results}

In 2020 and 2021, the results were pretty clear:

% Table generated by Excel2LaTeX from sheet '2020'
\begin{table}[H]
  \caption{Table generated by Excel2LaTeX from sheet 2020}
    \begin{tabular}{|l | r | r | r | r| r | r |}
    \hline
    CNI   & CPU & Memory & P2P & P2E & Avg & Rank \\
    \hline\hline
    Weight & 0.2   & 0.3   & 0.25  & 0.25  &       &  \\
    \hline
    Flannel & 4     & 4     & 4     & 4     & 4     & 1 \\
    \hline
    Calico & 3     & 2     & 2     & 2     & 2.2   & 2 \\
    \hline
    Canal & 2     & 3     & 1     & 1     & 1.8   & 4 \\
    \hline
    Cilium & 1     & 1     & 3     & 3     & 2     & 3 \\
    \hline
    \end{tabular}%
  \label{tab:e2l-2020}%
\end{table}%

% Table generated by Excel2LaTeX from sheet '2021'
\begin{table}[H]
  \caption{Table generated by Excel2LaTeX from sheet 2021}
    \begin{tabular}{|l | r | r | r | r| r | r |}
    \hline
    CNI   & CPU & Memory & P2P & P2E & Avg & Rank \\
    \hline\hline
    Weight & 0.2   & 0.3   & 0.25  & 0.25  &       &  \\
    \hline
    Flannel & 4     & 4     & 4     & 4     & 4     & 1 \\
    \hline
    Calico & 3     & 2     & 2     & 2     & 2.2   & 2 \\
    \hline
    Canal & 2     & 3     & 1     & 1     & 1.8   & 4 \\
    \hline
    Cilium & 1     & 1     & 3     & 3     & 2     & 3 \\
    \hline
    \end{tabular}%
  \label{tab:e2l-2021}%
\end{table}%

Flannel comes out on top, delivering the highest bandwidth with the lowest CPU and memory usage; Canal comes in second, and Cilium third. It is interesting to note that Canal takes last place, even though it is, in essence, a combination of Calico and Flannel. However, it seems to be weighed down by higher CPU and memory usage. 

In 2024, though, the faster hardware and the years of development and improvement of eBPF and Cilium are changing the picture. Cilium shows a lot of bandwidth gains and moves up to second place, past Calico and Canal.

% Table generated by Excel2LaTeX from sheet '2024'
\begin{table}[htbp]
  \caption{Table generated by Excel2LaTeX from sheet 2024}
    \begin{tabular}{|l | r | r | r | r| r | r |}
    \hline
    CNI   & CPU & Memory & P2P & P2E & Avg & Rank \\
    \hline\hline
    Weight & 0.2   & 0.3   & 0.25  & 0.25  &       &  \\
    \hline
    Flannel & 4     & 4     & 3     & 3     & 3.5   & 1 \\
    \hline
    Calico & 3     & 2     & 2     & 2     & 2.2   & 3 \\
    \hline
    Canal & 2     & 3     & 1     & 1     & 1.8   & 4 \\
    \hline
    Cilium & 1     & 1     & 4     & 4     & 2.5   & 2 \\
    \hline
    \end{tabular}%
  \label{tab:e2l-2024}%
\end{table}%

Flannel takes first place in all measurements, making it an excellent choice for using it as the CNI in Kubernetes cluster deployments.

Looking at bandwidth alone, though, Cilium would have taken first place. Also, with all the additional features we haven't looked at, such as a built-in load balancer and the myriad of observability features, it does make a compelling use case for itself. However, it comes with significantly higher CPU usage and memory consumption compared to our overall winner, Flannel.

For real,\footnote{See \textit{Kelly, J. (2024)}: Gen-Z Slang Is Revolutionizing Work Jargon. \cite{genzSlang}} I see Flannel as the best option for a CNI unless you need to implement network policies, and I, for one, was thrilled when SUSE announced full support for Flannel in RKE2 earlier this year.\footnote{See \textit{Frank, C. (2024)}: RKE2: Flannel? Flannel! \cite{rke2Flannel}}

From the data analysis, we can now clearly guide you to select either Flannel or Cilium as the CNI for your RKE cluster. Your choice will heavily depend on whether you need the additional features that Cilium offers or whether you can live with Flannel's flat overlay network and enjoy its simplicity.

\subsection{Outlook}

EBPF is currently the most talked-about feature in the Linux kernel, offering the most exciting prospects for network performance improvements. Cilium was developed on top of eBPF and will significantly benefit from future enhancements to it. It does not take much imagination to see a significant further increase in bandwidth in the future.

On the other hand, Flannel is relatively simple and has a low overall overhead, so all performance improvements in the Linux kernel for IP networking outside of eBPF will also benefit Flannel's performance and bandwidth. We can also most likely look forward to an increase in bandwidth in the future.

Both CNIs, Flannel, and Cilium, will likely remain the top choices to select as CNI for Kubernetes clusters for the next few years.
