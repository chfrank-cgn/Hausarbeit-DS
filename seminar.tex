%
% Angepasste FOM Seminarvorlage (2024)
%
\documentclass[12pt,a4paper,listof=totoc,bibliography=totoc]{scrartcl}

\usepackage[english]{babel}			% englische Namen/Umlaute
\usepackage[utf8]{inputenc}	    	% Zeichensatzkodierung
\usepackage{silence}
 \WarningFilter{scrartcl}{Usage of package `fancyhdr'}
 \WarningFilter{scrartcl}{Usage of package `parskip'}
\usepackage{fancyhdr}
\usepackage{graphicx}               % Einbinden von Bildern
\usepackage[hidelinks]{hyperref}	% Klickbare Verweise und \autoref{label}
\usepackage[intoc]{nomencl}
\usepackage{setspace}
\usepackage{parskip}
\usepackage{caption}
\usepackage{float}
% \usepackage{listings}
\usepackage{geometry}
 \geometry{a4paper, left=40mm, right=20mm, top=40mm, bottom=20mm}
\renewcommand{\familydefault}{\sfdefault}
\renewcommand{\ttdefault}{pcr}
% \renewcommand{\lstlistlistingname}{List of Code Examples}
% \renewcommand{\lstlistingname}{Example}
\usepackage{float}
\floatstyle{plaintop}
\restylefloat{table}

%
%RStudio
%
\usepackage{color}
\usepackage{fancyvrb}
\usepackage{framed}
\DefineVerbatimEnvironment{Highlighting}{Verbatim}{commandchars=\\\{\}}
\definecolor{shadecolor}{RGB}{241,243,245}
\newenvironment{Shaded}{\begin{snugshade}}{\end{snugshade}}
\newcommand{\AttributeTok}[1]{\textcolor[rgb]{0.40,0.45,0.13}{#1}}
\newcommand{\FunctionTok}[1]{\textcolor[rgb]{0.28,0.35,0.67}{#1}}
\newcommand{\NormalTok}[1]{\textcolor[rgb]{0.00,0.23,0.31}{#1}}
\newcommand{\OtherTok}[1]{\textcolor[rgb]{0.00,0.23,0.31}{#1}}
\newcommand{\SpecialCharTok}[1]{\textcolor[rgb]{0.37,0.37,0.37}{#1}}
\newcommand{\StringTok}[1]{\textcolor[rgb]{0.13,0.47,0.30}{#1}}

% Bildueberschrift oben und rechtsbuendig
\captionsetup{labelfont=bf, textfont=bf}
\captionsetup{justification=raggedright,singlelinecheck=false}

% Blocksatz
\def\justify{%
  \rightskip=0pt
  \spaceskip=0pt
  \xspaceskip=0pt
  \relax
}

%
%	Hier werden Titel, Bearbeiter und das Datum eingetragen
%
\newcommand\svthema{Rancher CNI Usage Recommendation}
\newcommand\svperson{Christian Frank (\#473088)}
\newcommand\svdatum{\today}
\newcommand\lvname{Data Science \& Security Analytics}
\newcommand\lvtyp{SS 2024}
\newcommand\lvinst{FOM - Hochschule für Oekonomie \& Management}
\newcommand\lvbetr{Prof. Dr. Alexander Lutz}

\hypersetup{ % Thema und Author in die Meta-Daten der PDF
  pdftitle={\svthema}, 
  pdfauthor={Christian Frank},
  pdfsubject={Rancher CNI usage recommendation},
  pdfkeywords={Kubernetes, Network, Rancher, CNI, Bandwidth}
}

\begin{document}

% Titel
\title{ \huge\textbf{\svthema} }
\author{ {\svperson} \\ \svdatum }
\date{ \normalsize \centering \includegraphics[width=0.3\textwidth]{FOM}\\ {\lvname} \\ {\lvbetr} \\ {\lvinst} \\ {\lvtyp} }

% Seitennummer oben
\pagestyle{fancy}
\fancyhf{}
\fancyhf[ch]{\thepage}
\renewcommand\headrulewidth{0pt}

\maketitle
\thispagestyle{empty} % laesst die Seitennummer auf der Titelseite verschwinden
\pagenumbering{Roman}

\begin{abstract}
This paper will perform a secondary analysis of Kubernetes CNI performance data, focusing on the CNIs available in the Rancher Kubernetes Engine. It aims to make a usage recommendation to support the selection of the CNI when installing a new cluster.

\end{abstract}

\vfill
\begin{figure}[h]
    \centering
    \includegraphics[]{CC-BY}
\end{figure}

This work is licensed under the Creative Commons Attribution 4.0 International License. To view a copy of this license, visit http://creativecommons.org/licenses/by/4.0/ or send a letter to Creative Commons, PO Box 1866, Mountain View, CA 94042, USA.

\cleardoublepage

\tableofcontents			% Inhaltsverzeichnis
\cleardoublepage

\listoffigures				% Abbildungsverzeichnis
\cleardoublepage

\listoftables               % Tabellen
\cleardoublepage

% \lstlistoflistings			% Codeverzeichnis
% \cleardoublepage

%
% Abkuerzungsverzeichnis
%
\makenomenclature
\renewcommand{\nomname}{List of Abbreviations}

\nomenclature{\textbf{APA}}{American Psychological Association}
\nomenclature{\textbf{BGP}}{Border Gateway Protocol}
\nomenclature{\textbf{CNI}}{Container Network Interface}
\nomenclature{\textbf{CNCF}}{Cloud Native Computing Foundation}
\nomenclature{\textbf{CSV}}{Comma-Separated Values}
\nomenclature{\textbf{eBPF}}{extended Berkeley Packet Filter}
\nomenclature{\textbf{EDA}}{Exploratory Data Analysis}
\nomenclature{\textbf{K8s}}{Kubernetes}
\nomenclature{\textbf{LAN}}{Local Area Network}
\nomenclature{\textbf{NIST}}{National Institute of Standards and Technology}
\nomenclature{\textbf{RKE}}{Rancher Kubernetes Engine}
\nomenclature{\textbf{TSV}}{Tab-Separated Values}
\nomenclature{\textbf{VXLAN}}{Virtual Extensible LAN}

\printnomenclature[1.5in]          % Abkuerzungsverzeichnis
\cleardoublepage

\pagenumbering{arabic}
\setcounter{page}{5}

%
%	Einfuehrung
%

\pagebreak
\section{Introduction}

\onehalfspacing

\subsection{Original Article}

In April 2024, \href{https://www.linkedin.com/in/alexisducastel/}{Alexis Ducastel} of \href{https://infrabuilder.com/}{InfraBuilder} published his benchmark results for Kubernetes network plugins.\footnote{See \textit{Ducastel, A. (2024)}: Benchmark results of Kubernetes network plugins. \cite{originalArticle}} This article continues a series of published benchmarks they had published in the years before. I obtained permission from Alexis Ducastel to use the historical benchmark data to explore the evolution of Kubernetes networking over time.

\subsection{Kubernetes}

Kubernetes, or K8s, is an open-source system designed to automate deploying, scaling, and managing applications built using containers. Containers package software in a standardized unit that includes all dependencies the software needs to run, like code, libraries, and settings. This makes them portable and efficient.

Kubernetes helps manage these containers by grouping them logically. This makes it easier to track and manage complex applications with many containers. The original inspiration for Kubernetes came from Google's internal container orchestration system, Borg.\footnote{See \textit{Gemini (2024)}: What is Kubernetes. \cite{bardKubernetes}} 

In 2015, Kubernetes reached the 1.0 milestone, and in 2016, it was donated to the CNCF; the current release of Kubernetes is 1.30.

"For the people who built it, for the people who release it, and for the furries who keep all of our clusters online, we present to you Kubernetes v1.30: Uwubernetes, the cutest release to date."\footnote{\textit{Dsouza, A. (2024)}: Kubernetes 1.30. \cite{uwubernetes}}

\begin{figure}[H]
\centering
\caption {Kubernetes 1.30 Release Logo}
\includegraphics[width=0.3\linewidth]{images/k8s-1.30.png}
\label{fig:uwubernetes}
\end{figure}

\subsection{Container Network Interfaces}

CNI plugins are essential components of Kubernetes clusters and are responsible for managing network connectivity between pods. They provide the underlying infrastructure for pod communication within and outside the cluster. Kubernetes offers a variety of CNI plugins, each with distinct features and performance characteristics, allowing administrators to select the optimal solution based on specific cluster requirements.\footnote{See \textit{Kubernetes (2024)}: Network Plugins. \cite{networkPlugin}}

In this paper, we will focus on the four CNI plugins available for the Rancher Kubernetes Engine:\footnote{See \textit{SUSE (2024)}: Network Options. \cite{networkOptions}}

\begin{itemize}
    \item \href{https://github.com/flannel-io/flannel}{Flannel}
    \item \href{https://www.tigera.io/project-calico/}{Calico}
    \item \href{https://docs.tigera.io/calico/latest/getting-started/kubernetes/flannel/install-for-flannel#installing-calico-for-policy-and-flannel-aka-canal-for-networking}{Canal}
    \item \href{https://github.com/cilium/cilium}{Cilium}
\end{itemize}

\subsubsection{Flannel}

Flannel is the oldest CNI in the list and is a well-established overlay network. It provides a Layer 3 network fabric for Kubernetes clusters. The simple and flat nature of the overlay network allows for easy troubleshooting. Its most commonly used transport backend is VXLAN.\footnote{See \textit{Frank, C. (2020)}: Behind the scenes of Flannel. \cite{flannel}}

\subsubsection{Calico}

Calico by \href{https://www.tigera.io/}{Tigera} uses IP routing and iptables for its data path and can create separate networks for various workloads. Calico is not a flat network, so it uses BGP to establish the routes between the nodes in a given Kubernetes cluster. Calico provides network policies and supports a variety of data planes.

\subsubsection{Canal}

Canal, also by Tigera, combines the Calico routing and policy engine with the Flannel transport. For RKE, Canal is the default CNI as it offers a VXLAN transport backend and network policies.

\subsubsection{Cilium}

Cilium by \href{https://isovalent.com/}{Isovalent} and now \href{https://www.cisco.com/}{Cisco} is the newest entry in the list of available CNIs for RKE. Cilium uses a data plane based on \href{https://ebpf.io/}{eBPF} and focuses on large networks and high network throughput.

\subsection{Research Question}

This paper will use data exploration techniques to guide which CNI to select for RKE2 based on current and historical performance data.\footnote{See \textit{Tukey, J.W. (1977)}: Exploratory data analysis. \cite{exploratoryDA}} For guidance, we will focus on throughput, CPU, and memory consumption.

\subsection{Gender-neutral Pronouns}

Our society is becoming more open, inclusive, and gender-fluid, and now I think it's time to think about using gender-neutral pronouns in scientific texts, too. Two well-known researchers, Abigail C. Saguy and Juliet A. Williams, both from UCLA, propose to use the singular they/them instead: "The universal singular they is inclusive of people who identify as male, female or nonbinary."\footnote{\textit{Saguy, A. (2020)}: Why We Should All Use They/Them Pronouns. \cite{pronouns}} The aim is to support an inclusive approach in science through gender-neutral language. 

In this paper, we'll attempt to follow this suggestion and invite all my readers to do the same for future articles. Thank you!

If you're not sure about the definitions of gender and sex and how to use them, have a look at the definitions\footnote{See \textit{APA (2021)}: Definitions Related to Sexual Orientation. \cite{apaDefinitions}} by the American Psychological Association.

\subsection{Climate Emergency}

As Professor Rahmstorf puts it: "Without immediate, decisive climate protection measures, my children currently attending high school could already experience a 3-degree warmer Earth. No one can say exactly what this world would look like—it would be too far outside the entire experience of human history. But almost certainly, this earth would be full of horrors for the people who would have to experience it."\footnote{\textit{Rahmstorf, A. (2024)}: Climate and Weather at 3 Degrees More. \cite{3dgreesMore}}


%
%	Begrifflichkeiten
%

\pagebreak
\section{Data Sources and Research Methods}

\onehalfspacing

\subsection{The Data}

For the data analysis and visualization in this chapter, I used CSV files based on the data from the original author's \href{https://github.com/InfraBuilder}{Github}.

\subsection{Data Wrangling}

\subsubsection{Header}

\subsubsection{Data Transformation}

\subsection{What's Not in the Data}

\subsection{Data Exploration}

\subsection{Tools}


%
%	Theorieteil
%

\pagebreak
\section{Data Exploration}

\onehalfspacing

\subsection{Data Layout}

The data consists of twelve datasets, one per year (3) and one per CNI (4).

After our data wrangling, all datasets now contain the following five blocks of metrics in a single row per observation:

\begin{itemize}
    \item Idle server memory and CPU consumption
    \item TCP Pod-to-Pod bandwidth, memory, and CPU consumption
    \item UDP Pod-to-Pod bandwidth, memory, and CPU consumption
    \item TCP Pod-to-Server bandwidth, memory, and CPU consumption
    \item UDP Pod-to-Server bandwidth, memory, and CPU consumption
\end{itemize}

For our data, Pod-to-Pod traffic refers to traffic between Pods in the same cluster, and Pod-to-Server traffic refers to traffic from a Pod to an outside application.

The individual field names are derived from each dataset's first row and are consistent across all sets.

\subsection{2020}

\subsubsection{Flannel}

The first data set we will look at is the 2020 data set for Flannel,
which we will read using R's built-in read.csv function.

\begin{Shaded}
\begin{Highlighting}[]
\NormalTok{main }\OtherTok{\textless{}{-}} \StringTok{"."}
\NormalTok{cni }\OtherTok{\textless{}{-}} \FunctionTok{read.csv}\NormalTok{(}\FunctionTok{here}\NormalTok{(main,}\StringTok{"data"}\NormalTok{,}\StringTok{"2020{-}flannel.csv"}\NormalTok{))}
\end{Highlighting}
\end{Shaded}

In the next step, we will have a look at the structure of our data and
visually inspect the measurements.

\begin{Shaded}
\begin{Highlighting}[]
\FunctionTok{str}\NormalTok{(cni)}
\end{Highlighting}
\end{Shaded}

\begin{verbatim}
'data.frame':   3095 obs. of  24 variables:
 $ smem  : num  593 596 590 591 588 ...
 $ scpu  : num  0.682 0.679 0.662 0.642 0.728 ...
 $ cmem  : num  588 595 591 587 592 ...
 $ ccpu  : num  0.572 0.601 0.54 0.517 0.579 ...
 $ tcpbw : num  9675 9757 9693 9689 9773 ...
 $ tcpsm : num  590 589 593 592 593 ...
 $ tcpsc : num  5.16 5.16 5.18 5.2 5.12 ...
 $ tcpcm : num  593 591 591 593 593 ...
 $ tcpcc : num  5.17 5.1 5.02 5.04 5.23 ...
 $ udpbw : num  9845 9842 9842 9841 9840 ...
 $ udpsm : num  591 590 592 594 589 ...
 $ udpsc : num  5.49 5.47 5.44 5.42 5.5 ...
 $ udpcm : num  608 611 608 613 614 ...
 $ udpcc : num  12.9 12.9 12.9 12.9 13 ...
 $ tcpebw: num  9831 9828 9827 9826 9838 ...
 $ tcpesm: num  589 596 591 590 589 ...
 $ tcpesc: num  4.92 5.07 5.08 5.03 5.14 ...
 $ tcpecm: num  595 601 595 595 593 ...
 $ tcpecc: num  5.47 5.55 5.46 5.52 5.29 ...
 $ udpebw: num  9823 9833 9818 9818 9833 ...
 $ udpesm: num  594 596 594 593 597 ...
 $ udpesc: num  5.25 5.3 5.34 5.41 5.23 ...
 $ udpecm: num  601 601 603 604 600 ...
 $ udpecc: num  12.6 12.7 12.7 12.7 12.6 ...
\end{verbatim}

The various values for memory consumption (MB), CPU usage (Percentage), and
bandwidth (MBit/s) look fine at first glance.

For further analysis, we will inspect the values a bit more
closely, starting with memory consumption. First, we'll look at the idle values and get:

\begin{Shaded}
\begin{Highlighting}[]
\FunctionTok{median}\NormalTok{(cni}\SpecialCharTok{\$}\NormalTok{smem)}
\end{Highlighting}
\end{Shaded}

\begin{verbatim}
[1] 592.8159
\end{verbatim}

\begin{Shaded}
\begin{Highlighting}[]
\FunctionTok{mean}\NormalTok{(cni}\SpecialCharTok{\$}\NormalTok{smem)}
\end{Highlighting}
\end{Shaded}

\begin{verbatim}
[1] 593.1625
\end{verbatim}

\begin{Shaded}
\begin{Highlighting}[]
\FunctionTok{min}\NormalTok{(cni}\SpecialCharTok{\$}\NormalTok{smem)}
\end{Highlighting}
\end{Shaded}

\begin{verbatim}
[1] 585
\end{verbatim}

\begin{Shaded}
\begin{Highlighting}[]
\FunctionTok{max}\NormalTok{(cni}\SpecialCharTok{\$}\NormalTok{smem)}
\end{Highlighting}
\end{Shaded}

\begin{verbatim}
[1] 606
\end{verbatim}

As the Median is more robust against outliers and works better with data that might be skewed, we'll focus on the median values going forward.\footnote{See \textit{Orn, A. (2023)}: Means and Medians: When To Use Which. \cite{meansMedians}}

Comparing the idle memory consumption with the other measurements for memory consumption (TCP and UDP, and Pod-to-Pod and Pod-to-Server), we get a fairly consistent picture:

\begin{Shaded}
\begin{Highlighting}[]
\FunctionTok{median}\NormalTok{(cni}\SpecialCharTok{\$}\NormalTok{tcpsm)}
\end{Highlighting}
\end{Shaded}

\begin{verbatim}
[1] 591.2724
\end{verbatim}

\begin{Shaded}
\begin{Highlighting}[]
\FunctionTok{median}\NormalTok{(cni}\SpecialCharTok{\$}\NormalTok{udpsm)}
\end{Highlighting}
\end{Shaded}

\begin{verbatim}
[1] 591.5915
\end{verbatim}

\begin{Shaded}
\begin{Highlighting}[]
\FunctionTok{median}\NormalTok{(cni}\SpecialCharTok{\$}\NormalTok{tcpesm)}
\end{Highlighting}
\end{Shaded}

\begin{verbatim}
[1] 590.1462
\end{verbatim}

\begin{Shaded}
\begin{Highlighting}[]
\FunctionTok{median}\NormalTok{(cni}\SpecialCharTok{\$}\NormalTok{udpesm)}
\end{Highlighting}
\end{Shaded}

\begin{verbatim}
[1] 594.3655
\end{verbatim}

Next, we will check CPU consumption, starting again with idle values:

\begin{Shaded}
\begin{Highlighting}[]
\FunctionTok{median}\NormalTok{(cni}\SpecialCharTok{\$}\NormalTok{scpu)}
\end{Highlighting}
\end{Shaded}

\begin{verbatim}
[1] 0.6546849
\end{verbatim}

\begin{Shaded}
\begin{Highlighting}[]
\FunctionTok{mean}\NormalTok{(cni}\SpecialCharTok{\$}\NormalTok{scpu)}
\end{Highlighting}
\end{Shaded}

\begin{verbatim}
[1] 0.6695833
\end{verbatim}

\begin{Shaded}
\begin{Highlighting}[]
\FunctionTok{min}\NormalTok{(cni}\SpecialCharTok{\$}\NormalTok{scpu)}
\end{Highlighting}
\end{Shaded}

\begin{verbatim}
[1] 0.6
\end{verbatim}

\begin{Shaded}
\begin{Highlighting}[]
\FunctionTok{max}\NormalTok{(cni}\SpecialCharTok{\$}\NormalTok{scpu)}
\end{Highlighting}
\end{Shaded}

\begin{verbatim}
[1] 0.82
\end{verbatim}

When we validate this with the other measurements for CPU consumption, we get again fairly consistent values across the measurements but a clear differentiation
to the idle values:

\begin{Shaded}
\begin{Highlighting}[]
\FunctionTok{median}\NormalTok{(cni}\SpecialCharTok{\$}\NormalTok{tcpsc)}
\end{Highlighting}
\end{Shaded}

\begin{verbatim}
[1] 5.162688
\end{verbatim}

\begin{Shaded}
\begin{Highlighting}[]
\FunctionTok{median}\NormalTok{(cni}\SpecialCharTok{\$}\NormalTok{udpsc)}
\end{Highlighting}
\end{Shaded}

\begin{verbatim}
[1] 5.483832
\end{verbatim}

\begin{Shaded}
\begin{Highlighting}[]
\FunctionTok{median}\NormalTok{(cni}\SpecialCharTok{\$}\NormalTok{tcpesc)}
\end{Highlighting}
\end{Shaded}

\begin{verbatim}
[1] 5.055657
\end{verbatim}

\begin{Shaded}
\begin{Highlighting}[]
\FunctionTok{median}\NormalTok{(cni}\SpecialCharTok{\$}\NormalTok{udpesc)}
\end{Highlighting}
\end{Shaded}

\begin{verbatim}
[1] 5.30168
\end{verbatim}

Idle CPU consumption does not seem to be a good overall performance indicator

The final data point we want to look at is bandwidth, the core
performance metric for a network:\footnote{See \textit{Solarwinds (2024)}: What are Network Performance Metrics? \cite{networkPerformance}}

\begin{Shaded}
\begin{Highlighting}[]
\FunctionTok{median}\NormalTok{(cni}\SpecialCharTok{\$}\NormalTok{tcpbw)}
\end{Highlighting}
\end{Shaded}

\begin{verbatim}
[1] 9705.809
\end{verbatim}

\begin{Shaded}
\begin{Highlighting}[]
\FunctionTok{median}\NormalTok{(cni}\SpecialCharTok{\$}\NormalTok{udpbw)}
\end{Highlighting}
\end{Shaded}

\begin{verbatim}
[1] 9843.559
\end{verbatim}

\begin{Shaded}
\begin{Highlighting}[]
\FunctionTok{median}\NormalTok{(cni}\SpecialCharTok{\$}\NormalTok{tcpebw)}
\end{Highlighting}
\end{Shaded}

\begin{verbatim}
[1] 9828.35
\end{verbatim}

\begin{Shaded}
\begin{Highlighting}[]
\FunctionTok{median}\NormalTok{(cni}\SpecialCharTok{\$}\NormalTok{udpebw)}
\end{Highlighting}
\end{Shaded}

\begin{verbatim}
[1] 9822.046
\end{verbatim}

With the majority of the internet traffic being TCP, we will focus on
the measurements for TCP for analysis and comparison.\footnote{See \textit{Quian, L. (2012)}: A flow-based performance analysis of TCP and TCP applications. \cite{tcpTraffic}}

We visualize both the TCP Pod-to-Pod and Pod-to-Server traffic bandwidth using histograms:

\begin{Shaded}
\begin{Highlighting}[]
\FunctionTok{gf\_histogram}\NormalTok{(}\SpecialCharTok{\textasciitilde{}}\NormalTok{tcpbw, }\AttributeTok{data =}\NormalTok{ cni")} 
\end{Highlighting}
\end{Shaded}

\begin{figure}[H]
\centering
\caption {Flannel Pod-to-Pod Bandwidth}
\includegraphics[width=\linewidth]{images/unnamed-chunk-8-1.png}
\label{fig:flannel-8-1}
\end{figure}

\begin{Shaded}
\begin{Highlighting}[]
\FunctionTok{gf\_histogram}\NormalTok{(}\SpecialCharTok{\textasciitilde{}}\NormalTok{tcpebw, }\AttributeTok{data =}\NormalTok{ cni")} 
\end{Highlighting}
\end{Shaded}

\begin{figure}[H]
\centering
\caption {Flannel Pod-to-Server Bandwidth}
\includegraphics[width=\linewidth]{images/unnamed-chunk-9-1.png}
\label{fig:flannel-9-1}
\end{figure}

The data distribution supports our choice to use the median values as key indicators for TCP bandwidth.

Following the results of a scientific coin toss,\footnote{See \textit{Kim, S.E. (2024)}: Scientists Destroy Illusion That Coin Toss Flips Are 50–50. \cite{coinToss}} we will use the TCP Pod-to Server values for CPU usage and memory consumption for further analysis.

From the 2020 Flannel dataset, we will thus use the following values:

\begin{Shaded}
\begin{Highlighting}[]
\FunctionTok{median}\NormalTok{(cni}\SpecialCharTok{\$}\NormalTok{tcpesm)}
\end{Highlighting}
\end{Shaded}

\begin{verbatim}
[1] 590.1462
\end{verbatim}

\begin{Shaded}
\begin{Highlighting}[]
\FunctionTok{median}\NormalTok{(cni}\SpecialCharTok{\$}\NormalTok{tcpesc)}
\end{Highlighting}
\end{Shaded}

\begin{verbatim}
[1] 5.055657
\end{verbatim}

\begin{Shaded}
\begin{Highlighting}[]
\FunctionTok{median}\NormalTok{(cni}\SpecialCharTok{\$}\NormalTok{tcpbw)}
\end{Highlighting}
\end{Shaded}

\begin{verbatim}
[1] 9705.809
\end{verbatim}

\begin{Shaded}
\begin{Highlighting}[]
\FunctionTok{median}\NormalTok{(cni}\SpecialCharTok{\$}\NormalTok{tcpebw)}
\end{Highlighting}
\end{Shaded}

\begin{verbatim}
[1] 9828.35
\end{verbatim}

\subsubsection{Calico}

We will now read the 2020 data set for Calico.

\begin{Shaded}
\begin{Highlighting}[]
\NormalTok{main }\OtherTok{\textless{}{-}} \StringTok{"."}
\NormalTok{cni }\OtherTok{\textless{}{-}} \FunctionTok{read.csv}\NormalTok{(}\FunctionTok{here}\NormalTok{(main,}\StringTok{"data"}\NormalTok{,}\StringTok{"2020{-}calico.csv"}\NormalTok{))}
\end{Highlighting}
\end{Shaded}

In the next step, we will examine the structure of our data and visually
inspect the measurements.

\begin{Shaded}
\begin{Highlighting}[]
\FunctionTok{str}\NormalTok{(cni)}
\end{Highlighting}
\end{Shaded}

\begin{verbatim}
'data.frame':   3091 obs. of  24 variables:
 $ smem  : num  667 665 666 669 665 ...
 $ scpu  : num  1.64 1.64 1.62 1.64 1.63 ...
 $ cmem  : num  666 666 666 653 666 ...
 $ ccpu  : num  1.27 1.27 1.31 1.3 1.29 ...
 $ tcpbw : num  8879 8885 8884 8882 8884 ...
 $ tcpsm : num  660 664 662 664 660 ...
 ...
\end{verbatim}

Then, we'll extract the key metrics for further analysis:

\begin{Shaded}
\begin{Highlighting}[]
\FunctionTok{median}\NormalTok{(cni}\SpecialCharTok{\$}\NormalTok{tcpesm)}
\end{Highlighting}
\end{Shaded}

\begin{verbatim}
[1] 659.3846
\end{verbatim}

\begin{Shaded}
\begin{Highlighting}[]
\FunctionTok{median}\NormalTok{(cni}\SpecialCharTok{\$}\NormalTok{tcpesc)}
\end{Highlighting}
\end{Shaded}

\begin{verbatim}
[1] 6.44727
\end{verbatim}

\begin{Shaded}
\begin{Highlighting}[]
\FunctionTok{median}\NormalTok{(cni}\SpecialCharTok{\$}\NormalTok{tcpbw)}
\end{Highlighting}
\end{Shaded}

\begin{verbatim}
[1] 8882.457
\end{verbatim}

\begin{Shaded}
\begin{Highlighting}[]
\FunctionTok{median}\NormalTok{(cni}\SpecialCharTok{\$}\NormalTok{tcpebw)}
\end{Highlighting}
\end{Shaded}

\begin{verbatim}
[1] 8675.868
\end{verbatim}

\subsubsection{Canal}

We will now read the 2020 data set for Canal.

\begin{Shaded}
\begin{Highlighting}[]
\NormalTok{main }\OtherTok{\textless{}{-}} \StringTok{"."}
\NormalTok{cni }\OtherTok{\textless{}{-}} \FunctionTok{read.csv}\NormalTok{(}\FunctionTok{here}\NormalTok{(main,}\StringTok{"data"}\NormalTok{,}\StringTok{"2020{-}canal.csv"}\NormalTok{))}
\end{Highlighting}
\end{Shaded}

In the next step, we will examine the structure of our data and visually
inspect the measurements.

\begin{Shaded}
\begin{Highlighting}[]
\FunctionTok{str}\NormalTok{(cni)}
\end{Highlighting}
\end{Shaded}

\begin{verbatim}
'data.frame':   3076 obs. of  24 variables:
 $ smem  : num  662 659 663 663 663 ...
 $ scpu  : num  1.43 1.56 1.49 1.47 1.43 ...
 $ cmem  : num  663 663 660 666 659 ...
 $ ccpu  : num  1.97 2.01 1.98 1.99 1.96 ...
 $ tcpbw : num  8490 8586 8635 8647 8648 ...
 $ tcpsm : num  659 656 655 657 657 ...
 ...
\end{verbatim}

Then, we'll extract the key metrics for further analysis:

\begin{Shaded}
\begin{Highlighting}[]
\FunctionTok{median}\NormalTok{(cni}\SpecialCharTok{\$}\NormalTok{tcpesm)}
\end{Highlighting}
\end{Shaded}

\begin{verbatim}
[1] 655.4456
\end{verbatim}

\begin{Shaded}
\begin{Highlighting}[]
\FunctionTok{median}\NormalTok{(cni}\SpecialCharTok{\$}\NormalTok{tcpesc)}
\end{Highlighting}
\end{Shaded}

\begin{verbatim}
[1] 6.697056
\end{verbatim}

\begin{Shaded}
\begin{Highlighting}[]
\FunctionTok{median}\NormalTok{(cni}\SpecialCharTok{\$}\NormalTok{tcpbw)}
\end{Highlighting}
\end{Shaded}

\begin{verbatim}
[1] 8634.413
\end{verbatim}

\begin{Shaded}
\begin{Highlighting}[]
\FunctionTok{median}\NormalTok{(cni}\SpecialCharTok{\$}\NormalTok{tcpebw)}
\end{Highlighting}
\end{Shaded}

\begin{verbatim}
[1] 8576.709
\end{verbatim}

\subsubsection{Cilium}

We will now read the 2020 data set for Cilium.

\begin{Shaded}
\begin{Highlighting}[]
\NormalTok{main }\OtherTok{\textless{}{-}} \StringTok{"."}
\NormalTok{cni }\OtherTok{\textless{}{-}} \FunctionTok{read.csv}\NormalTok{(}\FunctionTok{here}\NormalTok{(main,}\StringTok{"data"}\NormalTok{,}\StringTok{"2020{-}cilium.csv"}\NormalTok{))}
\end{Highlighting}
\end{Shaded}

In the next step, we will examine the structure of our data and visually
inspect the measurements.

\begin{Shaded}
\begin{Highlighting}[]
\FunctionTok{str}\NormalTok{(cni)}
\end{Highlighting}
\end{Shaded}

\begin{verbatim}
'data.frame':   3082 obs. of  24 variables:
 $ smem  : num  859 867 863 857 863 ...
 $ scpu  : num  3.95 3.68 2.63 3.32 3.53 ...
 $ cmem  : num  867 872 871 868 864 ...
 $ ccpu  : num  1.78 1.74 1.69 1.73 1.76 ...
 $ tcpbw : num  9445 9547 9467 9449 9466 ...
 $ tcpsm : num  861 867 864 864 864 ...
 ...
\end{verbatim}

Then, we'll extract the key metrics for further analysis:

\begin{Shaded}
\begin{Highlighting}[]
\FunctionTok{median}\NormalTok{(cni}\SpecialCharTok{\$}\NormalTok{tcpesm)}
\end{Highlighting}
\end{Shaded}

\begin{verbatim}
[1] 866.5502
\end{verbatim}

\begin{Shaded}
\begin{Highlighting}[]
\FunctionTok{median}\NormalTok{(cni}\SpecialCharTok{\$}\NormalTok{tcpesc)}
\end{Highlighting}
\end{Shaded}

\begin{verbatim}
[1] 13.2222
\end{verbatim}

\begin{Shaded}
\begin{Highlighting}[]
\FunctionTok{median}\NormalTok{(cni}\SpecialCharTok{\$}\NormalTok{tcpbw)}
\end{Highlighting}
\end{Shaded}

\begin{verbatim}
[1] 9475.213
\end{verbatim}

\begin{Shaded}
\begin{Highlighting}[]
\FunctionTok{median}\NormalTok{(cni}\SpecialCharTok{\$}\NormalTok{tcpebw)}
\end{Highlighting}
\end{Shaded}

\begin{verbatim}
[1] 9673.348
\end{verbatim}

\subsection{2021}

\subsubsection{Flannel}

We will now read the 2021 data set for Flannel.

\begin{Shaded}
\begin{Highlighting}[]
\NormalTok{main }\OtherTok{\textless{}{-}} \StringTok{"."}
\NormalTok{cni }\OtherTok{\textless{}{-}} \FunctionTok{read.csv}\NormalTok{(}\FunctionTok{here}\NormalTok{(main,}\StringTok{"data"}\NormalTok{,}\StringTok{"2021{-}flannel.csv"}\NormalTok{))}
\end{Highlighting}
\end{Shaded}

In the next step, we will examine the structure of our data and visually
inspect the measurements.

\begin{Shaded}
\begin{Highlighting}[]
\FunctionTok{str}\NormalTok{(cni)}
\end{Highlighting}
\end{Shaded}

\begin{verbatim}
'data.frame':   3112 obs. of  24 variables:
 $ smem  : num  591 596 599 593 599 ...
 $ scpu  : num  0.67 0.665 0.699 0.654 0.699 ...
 $ cmem  : num  596 592 591 589 580 ...
 $ ccpu  : num  0.539 0.502 0.546 0.554 0.55 ...
 $ tcpbw : num  9657 9737 9680 9755 9689 ...
 ...
\end{verbatim}

Then, we'll extract the key metrics for further analysis:

\begin{Shaded}
\begin{Highlighting}[]
\FunctionTok{median}\NormalTok{(cni}\SpecialCharTok{\$}\NormalTok{tcpesm)}
\end{Highlighting}
\end{Shaded}

\begin{verbatim}
[1] 590.5901
\end{verbatim}

\begin{Shaded}
\begin{Highlighting}[]
\FunctionTok{median}\NormalTok{(cni}\SpecialCharTok{\$}\NormalTok{tcpesc)}
\end{Highlighting}
\end{Shaded}

\begin{verbatim}
[1] 5.015945
\end{verbatim}

\begin{Shaded}
\begin{Highlighting}[]
\FunctionTok{median}\NormalTok{(cni}\SpecialCharTok{\$}\NormalTok{tcpbw)}
\end{Highlighting}
\end{Shaded}

\begin{verbatim}
[1] 9695.797
\end{verbatim}

\begin{Shaded}
\begin{Highlighting}[]
\FunctionTok{median}\NormalTok{(cni}\SpecialCharTok{\$}\NormalTok{tcpebw)}
\end{Highlighting}
\end{Shaded}

\begin{verbatim}
[1] 9825.204
\end{verbatim}

The difference between the measurements in the 2020 and 2021 data sets is not that big; the data was taken only a couple of months apart, and the CNI versions were relatively close to each other.

\subsubsection{Calico}

We will now read the 2021 data set for Calico.

\begin{Shaded}
\begin{Highlighting}[]
\NormalTok{main }\OtherTok{\textless{}{-}} \StringTok{"."}
\NormalTok{cni }\OtherTok{\textless{}{-}} \FunctionTok{read.csv}\NormalTok{(}\FunctionTok{here}\NormalTok{(main,}\StringTok{"data"}\NormalTok{,}\StringTok{"2021{-}calico.csv"}\NormalTok{))}
\end{Highlighting}
\end{Shaded}

In the next step, we will examine the structure of our data and visually
inspect the measurements.

\begin{Shaded}
\begin{Highlighting}[]
\FunctionTok{str}\NormalTok{(cni)}
\end{Highlighting}
\end{Shaded}

\begin{verbatim}
'data.frame':   3103 obs. of  24 variables:
 $ smem  : num  663 669 663 665 667 ...
 $ scpu  : num  1.56 1.56 1.64 1.6 1.61 ...
 $ cmem  : num  666 664 665 669 667 ...
 $ ccpu  : num  1.25 1.24 1.25 1.27 1.24 ...
 $ tcpbw : num  8868 8875 8873 8867 8880 ...
 $ tcpsm : num  664 665 663 659 663 ...
 ...
\end{verbatim}

Then, we'll extract the key metrics for further analysis:

\begin{Shaded}
\begin{Highlighting}[]
\FunctionTok{median}\NormalTok{(cni}\SpecialCharTok{\$}\NormalTok{tcpesm)}
\end{Highlighting}
\end{Shaded}

\begin{verbatim}
[1] 661.9909
\end{verbatim}

\begin{Shaded}
\begin{Highlighting}[]
\FunctionTok{median}\NormalTok{(cni}\SpecialCharTok{\$}\NormalTok{tcpesc)}
\end{Highlighting}
\end{Shaded}

\begin{verbatim}
[1] 6.366529
\end{verbatim}

\begin{Shaded}
\begin{Highlighting}[]
\FunctionTok{median}\NormalTok{(cni}\SpecialCharTok{\$}\NormalTok{tcpbw)}
\end{Highlighting}
\end{Shaded}

\begin{verbatim}
[1] 8876.478
\end{verbatim}

\begin{Shaded}
\begin{Highlighting}[]
\FunctionTok{median}\NormalTok{(cni}\SpecialCharTok{\$}\NormalTok{tcpebw)}
\end{Highlighting}
\end{Shaded}

\begin{verbatim}
[1] 8763.889
\end{verbatim}

\subsubsection{Canal}

We will now read the 2021 data set for Canal.

\begin{Shaded}
\begin{Highlighting}[]
\NormalTok{main }\OtherTok{\textless{}{-}} \StringTok{"."}
\NormalTok{cni }\OtherTok{\textless{}{-}} \FunctionTok{read.csv}\NormalTok{(}\FunctionTok{here}\NormalTok{(main,}\StringTok{"data"}\NormalTok{,}\StringTok{"2021{-}canal.csv"}\NormalTok{))}
\end{Highlighting}
\end{Shaded}

In the next step, we will examine the structure of our data and visually
inspect the measurements.

\begin{Shaded}
\begin{Highlighting}[]
\FunctionTok{str}\NormalTok{(cni)}
\end{Highlighting}
\end{Shaded}

\begin{verbatim}
'data.frame':   3115 obs. of  24 variables:
 $ smem  : num  657 661 660 663 656 ...
 $ scpu  : num  1.49 1.51 1.47 1.51 1.51 ...
 $ cmem  : num  653 658 666 656 651 ...
 $ ccpu  : num  1.99 1.86 1.94 2.01 2.01 ...
 $ tcpbw : num  8638 8636 8627 8639 8605 ...
 $ tcpsm : num  652 653 659 659 656 ...
 ...
\end{verbatim}

Then, we'll extract the key metrics for further analysis:

\begin{Shaded}
\begin{Highlighting}[]
\FunctionTok{median}\NormalTok{(cni}\SpecialCharTok{\$}\NormalTok{tcpesm)}
\end{Highlighting}
\end{Shaded}

\begin{verbatim}
[1] 659.3958
\end{verbatim}

\begin{Shaded}
\begin{Highlighting}[]
\FunctionTok{median}\NormalTok{(cni}\SpecialCharTok{\$}\NormalTok{tcpesc)}
\end{Highlighting}
\end{Shaded}

\begin{verbatim}
[1] 6.66032
\end{verbatim}

\begin{Shaded}
\begin{Highlighting}[]
\FunctionTok{median}\NormalTok{(cni}\SpecialCharTok{\$}\NormalTok{tcpbw)}
\end{Highlighting}
\end{Shaded}

\begin{verbatim}
[1] 8612.275
\end{verbatim}

\begin{Shaded}
\begin{Highlighting}[]
\FunctionTok{median}\NormalTok{(cni}\SpecialCharTok{\$}\NormalTok{tcpebw)}
\end{Highlighting}
\end{Shaded}

\begin{verbatim}
[1] 8579.366
\end{verbatim}

\subsubsection{Cilium}

We will now read the 2021 data set for Cilium.

\begin{Shaded}
\begin{Highlighting}[]
\NormalTok{main }\OtherTok{\textless{}{-}} \StringTok{"."}
\NormalTok{cni }\OtherTok{\textless{}{-}} \FunctionTok{read.csv}\NormalTok{(}\FunctionTok{here}\NormalTok{(main,}\StringTok{"data"}\NormalTok{,}\StringTok{"2021{-}cilium.csv"}\NormalTok{))}
\end{Highlighting}
\end{Shaded}

In the next step, we will examine the structure of our data and visually
inspect the measurements.

\begin{Shaded}
\begin{Highlighting}[]
\FunctionTok{str}\NormalTok{(cni)}
\end{Highlighting}
\end{Shaded}

\begin{verbatim}
'data.frame':   3105 obs. of  24 variables:
 $ smem  : num  871 868 868 867 868 ...
 $ scpu  : num  3.25 2.85 3.31 3.3 3.67 ...
 $ cmem  : num  864 869 866 880 879 ...
 $ ccpu  : num  1.69 1.7 1.66 1.65 1.72 ...
 $ tcpbw : num  9450 9458 9440 9532 9450 ...
 $ tcpsm : num  865 863 863 866 868 ...
 ...
\end{verbatim}

Then, we'll extract the key metrics for further analysis:

\begin{Shaded}
\begin{Highlighting}[]
\FunctionTok{median}\NormalTok{(cni}\SpecialCharTok{\$}\NormalTok{tcpesm)}
\end{Highlighting}
\end{Shaded}

\begin{verbatim}
[1] 867.1192
\end{verbatim}

\begin{Shaded}
\begin{Highlighting}[]
\FunctionTok{median}\NormalTok{(cni}\SpecialCharTok{\$}\NormalTok{tcpesc)}
\end{Highlighting}
\end{Shaded}

\begin{verbatim}
[1] 12.77646
\end{verbatim}

\begin{Shaded}
\begin{Highlighting}[]
\FunctionTok{median}\NormalTok{(cni}\SpecialCharTok{\$}\NormalTok{tcpbw)}
\end{Highlighting}
\end{Shaded}

\begin{verbatim}
[1] 9444.988
\end{verbatim}

\begin{Shaded}
\begin{Highlighting}[]
\FunctionTok{median}\NormalTok{(cni}\SpecialCharTok{\$}\NormalTok{tcpebw)}
\end{Highlighting}
\end{Shaded}

\begin{verbatim}
[1] 9679.113
\end{verbatim}

\subsection{2024}

\subsubsection{Flannel}

We will now read the 2024 data set for Flannel.

\begin{Shaded}
\begin{Highlighting}[]
\NormalTok{main }\OtherTok{\textless{}{-}} \StringTok{"."}
\NormalTok{cni }\OtherTok{\textless{}{-}} \FunctionTok{read.csv}\NormalTok{(}\FunctionTok{here}\NormalTok{(main,}\StringTok{"data"}\NormalTok{,}\StringTok{"2024{-}flannel.csv"}\NormalTok{))}
\end{Highlighting}
\end{Shaded}

In the next step, we will examine the structure of our data and visually
inspect the measurements.

\begin{Shaded}
\begin{Highlighting}[]
\FunctionTok{str}\NormalTok{(cni)}
\end{Highlighting}
\end{Shaded}

\begin{verbatim}
'data.frame':   3104 obs. of  24 variables:
 $ smem  : num  1204 1210 1209 1208 1209 ...
 $ scpu  : num  0.0149 0.0145 0.0146 0.0154 0.0153 ...
 $ cmem  : num  1187 1199 1201 1222 1219 ...
 $ ccpu  : num  0.0125 0.0121 0.0122 0.0122 0.0121 ...
 $ tcpbw : num  18936 19076 19206 19056 19097 ...
 ...
\end{verbatim}

Then, we'll extract the key metrics for further analysis:

\begin{Shaded}
\begin{Highlighting}[]
\FunctionTok{median}\NormalTok{(cni}\SpecialCharTok{\$}\NormalTok{tcpesm)}
\end{Highlighting}
\end{Shaded}

\begin{verbatim}
[1] 1175.85
\end{verbatim}

\begin{Shaded}
\begin{Highlighting}[]
\FunctionTok{median}\NormalTok{(cni}\SpecialCharTok{\$}\NormalTok{tcpesc)}
\end{Highlighting}
\end{Shaded}

\begin{verbatim}
[1] 5.052353
\end{verbatim}

\begin{Shaded}
\begin{Highlighting}[]
\FunctionTok{median}\NormalTok{(cni}\SpecialCharTok{\$}\NormalTok{tcpbw)}
\end{Highlighting}
\end{Shaded}

\begin{verbatim}
[1] 19166.04
\end{verbatim}

\begin{Shaded}
\begin{Highlighting}[]
\FunctionTok{median}\NormalTok{(cni}\SpecialCharTok{\$}\NormalTok{tcpebw)}
\end{Highlighting}
\end{Shaded}

\begin{verbatim}
[1] 19942.3
\end{verbatim}

Compared to the previous years, we can see a significant increase in bandwidth. Between the 2021 and 2024 measurements were significant improvements in Kubernetes development and also the infrastructure setup now has 40GBit interface cards and switches.

\subsubsection{Calico}

We will now read the 2024 data set for Calico.

\begin{Shaded}
\begin{Highlighting}[]
\NormalTok{main }\OtherTok{\textless{}{-}} \StringTok{"."}
\NormalTok{cni }\OtherTok{\textless{}{-}} \FunctionTok{read.csv}\NormalTok{(}\FunctionTok{here}\NormalTok{(main,}\StringTok{"data"}\NormalTok{,}\StringTok{"2024{-}calico.csv"}\NormalTok{))}
\end{Highlighting}
\end{Shaded}

In the next step, we will examine the structure of our data and visually
inspect the measurements.

\begin{Shaded}
\begin{Highlighting}[]
\FunctionTok{str}\NormalTok{(cni)}
\end{Highlighting}
\end{Shaded}

\begin{verbatim}
'data.frame':   3098 obs. of  24 variables:
 $ smem  : num  1435 1413 1414 1402 1419 ...
 $ scpu  : num  0.0213 0.02 0.0212 0.0213 0.0206 ...
 $ cmem  : num  1377 1370 1361 1369 1364 ...
 $ ccpu  : num  0.0259 0.025 0.0243 0.0265 0.0252 ...
 $ tcpbw : num  18520 18603 18600 18549 18524 ...
 $ tcpsm : num  1388 1390 1397 1394 1399 ...
 ...
\end{verbatim}

Then, we'll extract the key metrics for further analysis:

\begin{Shaded}
\begin{Highlighting}[]
\FunctionTok{median}\NormalTok{(cni}\SpecialCharTok{\$}\NormalTok{tcpesm)}
\end{Highlighting}
\end{Shaded}

\begin{verbatim}
[1] 1420.843
\end{verbatim}

\begin{Shaded}
\begin{Highlighting}[]
\FunctionTok{median}\NormalTok{(cni}\SpecialCharTok{\$}\NormalTok{tcpesc)}
\end{Highlighting}
\end{Shaded}

\begin{verbatim}
[1] 5.446998
\end{verbatim}

\begin{Shaded}
\begin{Highlighting}[]
\FunctionTok{median}\NormalTok{(cni}\SpecialCharTok{\$}\NormalTok{tcpbw)}
\end{Highlighting}
\end{Shaded}

\begin{verbatim}
[1] 18571.56
\end{verbatim}

\begin{Shaded}
\begin{Highlighting}[]
\FunctionTok{median}\NormalTok{(cni}\SpecialCharTok{\$}\NormalTok{tcpebw)}
\end{Highlighting}
\end{Shaded}

\begin{verbatim}
[1] 19263.95
\end{verbatim}

\subsubsection{Canal}

We will now read the 2024 data set for Canal.

\begin{Shaded}
\begin{Highlighting}[]
\NormalTok{main }\OtherTok{\textless{}{-}} \StringTok{"."}
\NormalTok{cni }\OtherTok{\textless{}{-}} \FunctionTok{read.csv}\NormalTok{(}\FunctionTok{here}\NormalTok{(main,}\StringTok{"data"}\NormalTok{,}\StringTok{"2024{-}canal.csv"}\NormalTok{))}
\end{Highlighting}
\end{Shaded}

In the next step, we will examine the structure of our data and visually
inspect the measurements.

\begin{Shaded}
\begin{Highlighting}[]
\FunctionTok{str}\NormalTok{(cni)}
\end{Highlighting}
\end{Shaded}

\begin{verbatim}
'data.frame':   3088 obs. of  24 variables:
 $ smem  : num  1243 1269 1269 1269 1245 ...
 $ scpu  : num  0.0281 0.0281 0.0281 0.0281 0.0281 ...
 $ cmem  : num  1235 1229 1228 1238 1238 ...
 $ ccpu  : num  0.0233 0.0251 0.0228 0.0284 0.0276 ...
 $ tcpbw : num  16841 16843 16840 16845 16847 ...
 $ tcpsm : num  1260 1279 1257 1242 1233 ...
 ...
\end{verbatim}

Then, we'll extract the key metrics for further analysis:

\begin{Shaded}
\begin{Highlighting}[]
\FunctionTok{median}\NormalTok{(cni}\SpecialCharTok{\$}\NormalTok{tcpesm)}
\end{Highlighting}
\end{Shaded}

\begin{verbatim}
[1] 1285.42
\end{verbatim}

\begin{Shaded}
\begin{Highlighting}[]
\FunctionTok{median}\NormalTok{(cni}\SpecialCharTok{\$}\NormalTok{tcpesc)}
\end{Highlighting}
\end{Shaded}

\begin{verbatim}
[1] 5.647729
\end{verbatim}

\begin{Shaded}
\begin{Highlighting}[]
\FunctionTok{median}\NormalTok{(cni}\SpecialCharTok{\$}\NormalTok{tcpbw)}
\end{Highlighting}
\end{Shaded}

\begin{verbatim}
[1] 16842.78
\end{verbatim}

\begin{Shaded}
\begin{Highlighting}[]
\FunctionTok{median}\NormalTok{(cni}\SpecialCharTok{\$}\NormalTok{tcpebw)}
\end{Highlighting}
\end{Shaded}

\begin{verbatim}
[1] 16328.69
\end{verbatim}

\subsubsection{Cilium}

We will now read the 2024 data set for Cilium.

\begin{Shaded}
\begin{Highlighting}[]
\NormalTok{main }\OtherTok{\textless{}{-}} \StringTok{"."}
\NormalTok{cni }\OtherTok{\textless{}{-}} \FunctionTok{read.csv}\NormalTok{(}\FunctionTok{here}\NormalTok{(main,}\StringTok{"data"}\NormalTok{,}\StringTok{"2024{-}cilium.csv"}\NormalTok{))}
\end{Highlighting}
\end{Shaded}

In the next step, we will examine the structure of our data and visually
inspect the measurements.

\begin{Shaded}
\begin{Highlighting}[]
\FunctionTok{str}\NormalTok{(cni)}
\end{Highlighting}
\end{Shaded}

\begin{verbatim}
'data.frame':   3101 obs. of  24 variables:
 $ smem  : num  1612 1613 1607 1605 1604 ...
 $ scpu  : num  0.0111 0.0109 0.0111 0.0126 0.0121 ...
 $ cmem  : num  1596 1592 1603 1597 1594 ...
 $ ccpu  : num  0.0118 0.0122 0.0125 0.0123 0.0122 ...
 $ tcpbw : num  20929 20932 20761 20525 20675 ...
 $ tcpsm : num  1538 1538 1538 1538 1534 ...
 ...
\end{verbatim}

Then, we'll extract the key metrics for further analysis:

\begin{Shaded}
\begin{Highlighting}[]
\FunctionTok{median}\NormalTok{(cni}\SpecialCharTok{\$}\NormalTok{tcpesm)}
\end{Highlighting}
\end{Shaded}

\begin{verbatim}
[1] 1521.506
\end{verbatim}

\begin{Shaded}
\begin{Highlighting}[]
\FunctionTok{median}\NormalTok{(cni}\SpecialCharTok{\$}\NormalTok{tcpesc)}
\end{Highlighting}
\end{Shaded}

\begin{verbatim}
[1] 6.152466
\end{verbatim}

\begin{Shaded}
\begin{Highlighting}[]
\FunctionTok{median}\NormalTok{(cni}\SpecialCharTok{\$}\NormalTok{tcpbw)}
\end{Highlighting}
\end{Shaded}

\begin{verbatim}
[1] 20709.53
\end{verbatim}

\begin{Shaded}
\begin{Highlighting}[]
\FunctionTok{median}\NormalTok{(cni}\SpecialCharTok{\$}\NormalTok{tcpebw)}
\end{Highlighting}
\end{Shaded}

\begin{verbatim}
[1] 21758.27
\end{verbatim}

We have now processed all datasets.


%
%	Praxisbezug
%

\pagebreak
\section{Data Analysis}

\onehalfspacing

\subsection{Performance Considerations}

\subsection{CPU Usage}

\begin{table}[h!]
\caption{Median CPU Usage}
\begin{tabular}{||c | c | c | c | c||} 
 \hline
 Year & Flannel & Calico & Canal & Cilium \\
 \hline\hline
 2020 & 0 & 0 & 0 & 0 \\ 
 \hline
 2021 & 0 & 0 & 0 & 0 \\
 \hline
 2024 & 0 & 0 & 0 & 0 \\
 \hline
\end{tabular}
\label{tab:cpu}
\end{table}

\subsection{Memory Usage}

\begin{table}[h!]
\caption{Median Memory Consumption}
\begin{tabular}{||c | c | c | c | c||} 
 \hline
 Year & Flannel & Calico & Canal & Cilium \\
 \hline\hline
 2020 & 0 & 0 & 0 & 0 \\ 
 \hline
 2021 & 0 & 0 & 0 & 0 \\
 \hline
 2024 & 0 & 0 & 0 & 0 \\
 \hline
\end{tabular}
\label{tab:mem}
\end{table}

\subsection{Throughput}

\begin{table}[h!]
\caption{Median Bandwidth}
\begin{tabular}{||c | c | c | c | c||} 
 \hline
 Year & Flannel & Calico & Canal & Cilium \\
 \hline\hline
 2020 & 0 & 0 & 0 & 0 \\ 
 \hline
 2021 & 0 & 0 & 0 & 0 \\
 \hline
 2024 & 0 & 0 & 0 & 0 \\
 \hline
\end{tabular}
\label{tab:bw}
\end{table}

\subsection{Results}

\subsection{Outlook}


%
%	Fazit
%

\pagebreak
\section{Summary}

\onehalfspacing

From the analysis, we were able to show marked improvements in Kubernetes networking. We saw performance improvements across the board for the CNIs included in the Rancher Kubernetes Engine.

The power is in the data—we were able to identify the two CNIs that will deliver the best performance and recommend them for future cluster configurations.

Cilium is the most promising new development in Kubernetes networking, and the venerable Flannel is holding up well.

We were merely able to scratch the surface with this secondary analysis, but I do hope that you will find at least some valuable insights and pointers to start with; none of this would have been possible without the excellent work by Alexis Duscatel of infraBuilder!

To quote Toshinori Yagi: “Next, it’s your turn.”\footnote{\textit{Crunchyroll (2024)}: 40 My Hero Academia Quotes Worth Remembering. \cite{mhaQuotes}} - go and create your own cluster networks!

Happy Ranching!


% Literaturverzeichnis
\cleardoublepage
\raggedright
\bibliographystyle{IEEEtranS}	% ieeetran verwenden, damit auch URLs angezeigt werden
\bibliography{seminar-lit}

\cleardoublepage
\justify
\input{erklaerung.tex}

\end{document}
