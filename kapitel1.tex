%
%	Einfuehrung
%

\pagebreak
\section{Introduction}

\onehalfspacing

\subsection{Original Article}

In April 2024, \href{https://www.linkedin.com/in/alexisducastel/}{Alexis Ducastel} of \href{https://infrabuilder.com/}{InfraBuilder} published his benchmark results for Kubernetes network plugins.\footnote{See \textit{Ducastel, A. (2024)}: Benchmark results of Kubernetes network plugins. \cite{originalArticle}} This article continues a series of published benchmarks they had published in the years before. I obtained permission from Alexis Ducastel to use the historical benchmark data to explore the evolution of Kubernetes networking over time.

\subsection{Kubernetes}

Kubernetes, or K8s, is an open-source system designed to automate deploying, scaling, and managing applications built using containers. Containers package software in a standardized unit that includes all dependencies the software needs to run, like code, libraries, and settings. This makes them portable and efficient.

Kubernetes helps manage these containers by grouping them logically. This makes it easier to track and manage complex applications with many containers. The original inspiration for Kubernetes came from Google's internal container orchestration system, Borg.\footnote{See \textit{Gemini (2024)}: What is Kubernetes. \cite{bardKubernetes}} 

In 2015, Kubernetes reached the 1.0 milestone, and in 2016, it was donated to the CNCF; the current release of Kubernetes is 1.30.

"For the people who built it, for the people who release it, and for the furries who keep all of our clusters online, we present to you Kubernetes v1.30: Uwubernetes, the cutest release to date."\footnote{\textit{Dsouza, A. (2024)}: Kubernetes 1.30. \cite{uwubernetes}}

\begin{figure}[H]
\centering
\caption {Kubernetes 1.30 Release Logo}
\includegraphics[width=0.3\linewidth]{images/k8s-1.30.png}
\label{fig:uwubernetes}
\end{figure}

\subsection{Container Network Interfaces}

CNI plugins are essential components of Kubernetes clusters and are responsible for managing network connectivity between pods. They provide the underlying infrastructure for pod communication within and outside the cluster. Kubernetes offers a variety of CNI plugins, each with distinct features and performance characteristics, allowing administrators to select the optimal solution based on specific cluster requirements.\footnote{See \textit{Kubernetes (2024)}: Network Plugins. \cite{networkPlugin}}

In this paper, we will focus on the four CNI plugins available for the Rancher Kubernetes Engine:\footnote{See \textit{SUSE (2024)}: Network Options. \cite{networkOptions}}

\begin{itemize}
    \item \href{https://github.com/flannel-io/flannel}{Flannel}
    \item \href{https://www.tigera.io/project-calico/}{Calico}
    \item \href{https://docs.tigera.io/calico/latest/getting-started/kubernetes/flannel/install-for-flannel#installing-calico-for-policy-and-flannel-aka-canal-for-networking}{Canal}
    \item \href{https://github.com/cilium/cilium}{Cilium}
\end{itemize}

\subsubsection{Flannel}

Flannel is the oldest CNI in the list and is a well-established overlay network. It provides a Layer 3 network fabric for Kubernetes clusters. The simple and flat nature of the overlay network allows for easy troubleshooting. Its most commonly used transport backend is VXLAN.\footnote{See \textit{Frank, C. (2020)}: Behind the scenes of Flannel. \cite{flannel}}

\subsubsection{Calico}

Calico by \href{https://www.tigera.io/}{Tigera} uses IP routing and iptables for its data path and can create separate networks for various workloads. Calico is not a flat network, so it uses BGP to establish the routes between the nodes in a given Kubernetes cluster. Calico provides network policies and supports a variety of data planes.

\subsubsection{Canal}

Canal, also by Tigera, combines the Calico routing and policy engine with the Flannel transport. For RKE, Canal is the default CNI as it offers a VXLAN transport backend and network policies.

\subsubsection{Cilium}

Cilium by \href{https://isovalent.com/}{Isovalent} and now \href{https://www.cisco.com/}{Cisco} is the newest entry in the list of available CNIs for RKE. Cilium uses a data plane based on \href{https://ebpf.io/}{eBPF} and focuses on large networks and high network throughput.

\subsection{Research Question}

This paper will use data exploration techniques to guide which CNI to select for RKE2 based on current and historical performance data.\footnote{See \textit{Tukey, J.W. (1977)}: Exploratory data analysis. \cite{exploratoryDA}} For guidance, we will focus on throughput, CPU, and memory consumption.

\subsection{Gender-neutral Pronouns}

Our society is becoming more open, inclusive, and gender-fluid, and now I think it's time to think about using gender-neutral pronouns in scientific texts, too. Two well-known researchers, Abigail C. Saguy and Juliet A. Williams, both from UCLA, propose to use the singular they/them instead: "The universal singular they is inclusive of people who identify as male, female or nonbinary."\footnote{\textit{Saguy, A. (2020)}: Why We Should All Use They/Them Pronouns. \cite{pronouns}} The aim is to support an inclusive approach in science through gender-neutral language. 

In this paper, we'll attempt to follow this suggestion and invite all my readers to do the same for future articles. Thank you!

If you're not sure about the definitions of gender and sex and how to use them, have a look at the definitions\footnote{See \textit{APA (2021)}: Definitions Related to Sexual Orientation. \cite{apaDefinitions}} by the American Psychological Association.

\subsection{Climate Emergency}

As Professor Rahmstorf puts it: "Without immediate, decisive climate protection measures, my children currently attending high school could already experience a 3-degree warmer Earth. No one can say exactly what this world would look like—it would be too far outside the entire experience of human history. But almost certainly, this earth would be full of horrors for the people who would have to experience it."\footnote{\textit{Rahmstorf, A. (2024)}: Climate and Weather at 3 Degrees More. \cite{3dgreesMore}}
