%
%	Einfuehrung
%

\pagebreak
\section{Introduction}

\onehalfspacing

\subsection{Original Article}

Original Article.\footnote{See \textit{Ducastel, A. (2024)}: Benchmark results of Kubernetes network plugins. \cite{originalArticle}} 

\subsection{Kubernetes}

Kubernetes, or K8s, is an open-source system designed to automate deploying, scaling, and managing applications built using containers. Containers package software in a standardized unit that includes all dependencies the software needs to run, like code, libraries, and settings. This makes them portable and efficient.

Kubernetes helps manage these containers by grouping them logically. This makes it easier to track and manage complex applications with many containers. The original inspiration for Kubernetes came from Google's internal container orchestration system, Borg.\footnote{See \textit{Gemini (2024)}: What is Kubernetes. \cite{bardKubernetes}} 

In 2015, Kubernetes reached the 1.0 milestone, and in 2016, it was donated to the CNCF; the current release of Kubernetes is 1.30.

"For the people who built it, for the people who release it, and for the furries who keep all of our clusters online, we present to you Kubernetes v1.30: Uwubernetes, the cutest release to date."\footnote{\textit{Dsouza, A. (2024)}: Kubernetes 1.30. \cite{uwubernetes}}

\begin{figure}[H]
\centering
\caption {Kubernetes 1.30 Release Logo}
\includegraphics[width=0.3\linewidth]{images/k8s-1.30.png}
\label{fig:uwubernetes}
\end{figure}

\subsection{Container Network Interfaces}

\subsubsection{Flannel}

Flannel is a well-established overlay network and provides a pure Layer 3 network fabric for Kubernetes clusters.\footnote{See \textit{Frank, C. (2020)}: Behind the scenes of Flannel. \cite{flannel}}

\subsubsection{Calico}

\subsubsection{Canal}

\subsubsection{Cilium}

\subsection{Research Question}

\subsection{Gender-neutral Pronouns}

Our society is becoming more open, inclusive, and gender-fluid, and now I think it's time to think about using gender-neutral pronouns in scientific texts, too. Two well-known researchers, Abigail C. Saguy and Juliet A. Williams, both from UCLA, propose to use the singular they/them instead: "The universal singular they is inclusive of people who identify as male, female or nonbinary."\footnote{\textit{Saguy, A. (2020)}: Why We Should All Use They/Them Pronouns. \cite{pronouns}} The aim is to support an inclusive approach in science through gender-neutral language. 

In this paper, I'll attempt to follow this suggestion and invite all my readers to do the same for future articles. Thank you!

If you're not sure about the definitions of gender and sex and how to use them, have a look at the definitions\footnote{See \textit{APA (2021)}: Definitions Related to Sexual Orientation. \cite{apaDefinitions}} by the American Psychological Association.

\subsection{Caveat: Online Learning and Loneliness}

According to a recent study conducted by the American Psychological Association, social distancing and online learning over an extended period can increase loneliness and significantly affect people's health.\footnote{See \textit{Luchetti, M. (2020)}: The trajectory of loneliness in response to COVID-19. \cite{apaLoneliness}}
